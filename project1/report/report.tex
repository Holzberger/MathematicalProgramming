\documentclass[11pt]{article}
\input{Packages}


\title{Programming Exercise 1}
\author{e11921655 Fabian Holzberger}
\date{\today}

\begin{document}
\graphicspath{{./figures/}}
\maketitle

%
\section{Introduction}

\section{SCF-Formulation}
\subsection{Model}
In the directed SCF-Formulation we deal with arc variables $x_{i,j}$ defined by (\ref{scf:f8}), flow variables $f_{i,j}$ defined on each arc by (\ref{scf:f9}) and binary node variables $z_i$ defined by (\ref{scf:f10}) on each node. The directed SCF-Formulation is given as:
\begin{gather}
  \min\sum\limits_{\{i,j\}\in E}(x_{i,j}+ x_{j,i})c_{i,j} \label{scf:f0}\\
  \sum\limits_{(i,j)\in \delta^+(0) } x_{i,j} =1 \label{scf:f1}\\
  \sum\limits_{(i,j)\in \delta^+(0) } f_{i,j} =k \label{scf:f2}\\
  \sum\limits_{(i,j)\in \delta^-(0) } x_{i,j} =0 \label{scf:f3}\\
  \sum\limits_{(i,j)\in \delta^-(0) } f_{i,j} =0 \label{scf:f4}\\
  f_{i,j} \leq kx_{i,j} \quad \forall (i,j)\in A\label{scf:f5}\\
  \sum\limits_{j\in N} z_j = k+1\label{scf:f6}\\
  \sum\limits_{(i,j)\in \delta^-(l) } f_{i,j} - \sum\limits_{(i,j)\in \delta^+(l) } f_{i,j} = z_l \quad l\in N \setminus \{0\}\label{scf:f7}\\
  x_{i,j} \in \mathbb{B} \quad \forall (i,j)\in A\label{scf:f8}\\
  f_{i,j} \in \mathbb{N}\cap[0,k] \quad \forall (i,j)\in A\label{scf:f9}\\
  z_{i} \in \mathbb{B} \quad \forall i \in N\label{scf:f10}\\
\end{gather}
Here in (\ref{scf:f0}) the cost of all arcs is summed where by optimality the formulation will only allow to select one arc at each undirected esge. Next in (\ref{scf:f1}) we set the number of outgoing arcs from the dummy root node to exactly $1$ and similarly in constraint (\ref{scf:f2}) the outflow from the dummy root to exactly $k$. Since we dont want an ingoing edge into the dummy root and further no flow back into the root we state these restrictions in the constraints (\ref{scf:f3}) and (\ref{scf:f4}). Constraint (\ref{scf:f5}) states that whenever we have a nonzero flow we musst select the corresponding arc for that flow. Any selected node consumes one unit of flow such that the sum of the outgoing flow variables is one less than the sum of the ingoing flow variables, which is formulated in (\ref{scf:f7}). Note that the latter constraint is not formulated for the root node since there all incoming flows are $0$. 


\section{MCF-Formulation}
\subsection{Model}
We formulate a directd MCF-Formulation. It uses the binary arc variables $x_{i,j}$ defined in (\ref{mcf:f8}), binary node variables $z_i$ defined by (\ref{mcf:f10}) where we exclude the root node from this variables and the positive flow variables $f_{i,j}^\alpha$ for each arc and node (exept the dummy root) that are defined in (\ref{mcf:f9}). 
\begin{gather}
  \min\sum\limits_{\{i,j\}\in E}(x_{i,j}+ x_{j,i})c_{i,j} \label{mcf:f0}\\
  \sum\limits_{(i,j)\in \delta^+(0) } x_{i,j} =1 \label{mcf:f1}\\
  \sum\limits_{(i,j)\in \delta^+(0) } f_{i,j}^\alpha =z_\alpha \quad \forall \alpha\in N \setminus \{0\} \label{mcf:f2}\\
  \sum\limits_{(i,j)\in \delta^-(0) } x_{i,j} =0 \label{mcf:f3}\\
  \sum\limits_{(i,j)\in \delta^-(0) } f_{i,j}^\alpha =0 \quad \forall \alpha \in N \setminus \{0\} \label{mcf:f4}\\
  f_{i,j}^\alpha \leq x_{i,j} \quad \forall (i,j)\in A \ \forall \alpha \in N\setminus\{0\}\label{mcf:f5}\\
  \sum\limits_{j\in N\setminus \{0\}} z_j = k\label{mcf:f6}\\
  \sum\limits_{(i,j)\in \delta^-(l) } f_{i,j}^\alpha - \sum\limits_{(i,j)\in \delta^+(l) } f_{i,j}^\alpha = 
  \begin{cases} z_l \ \text{ if } \alpha = l \neq 0 \\ 0 \ \text{ if } \alpha \neq l \quad \forall \alpha, l \in N \setminus \{0\} \end{cases} \label{mcf:f7}\\
  x_{i,j} \in \mathbb{B} \quad \forall (i,j)\in A\label{mcf:f8}\\
  f_{i,j}^\alpha \in \mathbb{N}\cap[0,1] \quad \forall (i,j)\in A \ \forall \alpha \in N \setminus \{0\}\label{mcf:f9}\\
  z_{i} \in \mathbb{B} \quad \forall i \in N\setminus \{0\}\label{mcf:f10}\\
\end{gather}
As in the SCF-Formulation we select exactly one outgoing arc from the root and no ingoing arc by the constraints (\ref{mcf:f1}) and (\ref{mcf:f3}). For this formulation we can have for each node a different commodity of flow and therefore we only need to send out one unit of a commodity from the root that corresponds to a selected node as can be seen in constraint (\ref{mcf:f2}). Again we dont want any commodity of flow back into the root which is stated in (\ref{mcf:f4}).
By (\ref{mcf:f5}) we select an arc if it has a nonzero flow in any commodity. Snce we dont include the root in the node variables we only need to select $k$ of them in (\ref{mcf:f6}). Constraint (\ref{mcf:f7}) is stating that when we select a node variable, that node consumes its assigned commodity by none of the other commodities. 


\section{MTZ-Formulation}
\subsection{Model}
For a directed MTZ-formulation we have binary arc-variables $x_{i,j}$, defnined in (\ref{mtz:f7}), binary node-variables $z_{i}$, defined in (\ref{mtz:f8}) and bfs-like level-variables $f_i$ in (\ref{mtz:f9}), that are positive integers and mark the level of the tree starting with $0$ at the dummy root (\ref{mtz:f10}). Our MTZ-formulation is given by: 
\begin{gather}
  \min\sum\limits_{\{i,j\}\in E}(x_{i,j}+ x_{j,i})c_{i,j} \label{mtz:f0}\\
  \sum\limits_{(i,j)\in \delta^+(0) } x_{i,j} =1 \label{mtz:f1}\\
  \sum\limits_{(i,j)\in \delta^-(0) } x_{i,j} =0 \label{mtz:f2}\\
  \sum\limits_{j\in N} z_j = k\label{mtz:f3}\\
  f_i + x_{i,j} \leq f_j + k(1-x_{i,j}) \quad \forall (i,j)\in A \label{mtz:f4} \\
  \sum\limits_{(i,j)\in \delta^-(l)} x_{i,j} = z_l \quad \forall l\in N\label{mtz:f5}\\
  \sum\limits_{(i,j)\in \delta^+(l)} x_{i,j} \leq (k-1)z_l \quad \forall l\in N\setminus \{0\}\label{mtz:f6}\\
  x_{i,j} \in \mathbb{B} \quad \forall (i,j)\in A\label{mtz:f7}\\
  z_{i} \in \mathbb{B} \quad \forall i \in N\label{mtz:f8}\\
  f_{i} \in \mathbb{N}\cap[1,k-1] \quad \forall i \in N \setminus \{0\}\label{mtz:f9}\\
  f_{0} = 0 \label{mtz:f10}
\end{gather}
Since we need to deals with the dummy root we select by (\ref{mtz:f0}) the number of outgoing arcs from the root to exactly $1$ and in (\ref{mtz:f1}) the number of ingoing root-arcs to $0$.
By (\ref{mtz:f3}) we select exacly $k$ nodes, note that the other constraints force that the dummy root is not selected. 
In (\ref{mtz:f4}) the MTZ constraint forces in every selected arc the level variable to increase by $1$ when going from the start to the end-node of the arc. Next the arc and node variables are connected by (\ref{mtz:f5}), which tells us that for every selected node there musst be an ingoing arc and (\ref{mtz:f6}) that describes that there can only be an outgoing arc at some node when that node is selected. Note that the latter constraint is not formulated for the dummy root, which enables solvability. 
%
%Bibliography
\newpage
\bibliography{lib} 
\bibliographystyle{ieeetr}

\end{document}
